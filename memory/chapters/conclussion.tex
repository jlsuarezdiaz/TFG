\chapter{Conclusiones y vías futuras}

En este trabajo se ha estudiado el concepto de aprendizaje de métricas de distancia, aprendiendo para ello en qué consiste, cuáles son sus aplicaciones, cómo diseñar sus algoritmos, y los fundamentos teóricos de esta disciplina. También se han estudiado algunos de los algoritmos más populares en este ámbito, también con sus fundamentos teóricos, y descubriendo distintas técnicas de resolución.

Para entender los fundamentos teóricos del aprendizaje de métricas de distancia y sus algoritmos, ha sido necesario profundizar en tres teorías matemáticas diferentes: el análisis convexo, el análisis matricial y la teoría de la información. El análisis convexo ha permitido presentar muchos de los problemas de optimización estudiados en los algoritmos, junto con algunos métodos para resolverlos. El análisis matricial ha proporcionado gran cantidad de resultados útiles para comprender esta disciplina, desde cómo parametrizar distancias de Mahalanobis, hasta la optimización de funciones utilizando vectores propios, pasando por el algoritmo más básico de programación semidefinida. Por último, la teoría de la información ha motivado varios de los algoritmos que hemos estudiado.

En cuanto al software desarrollado, se ha conseguido integrar los algoritmos de aprendizaje de métricas de distancia en una biblioteca elaborada en el lenguaje Python, que además dispone de una interfaz para el lenguaje R, con funcionalidades adicionales como clasificadores, visualización o estimación de parámetros. 

Además, se han desarrollado varios experimentos que han permitido evaluar la capacidad de los distintos algoritmos. Los resultados de estos experimentos nos han permitido observar cómo algoritmos como LMNN, DMLMJ, y especialmente NCA pueden mejorar considerablemente la clasificación por vecinos cercanos y cómo los algoritmos de aprendizaje basados en centroides mejoran también a sus correspondientes clasificadores. También hemos podido comprobar la gran variedad de posibilidades que ofrecen los algoritmos basados en kernels y las ventajas que puede ofrecer una reducción adecuada de la dimensión del conjunto de datos.

Los objetivos iniciales propuestos se han cumplido, aunque el trabajo elaborado ha abierto el camino a varias vías de avance, tanto en el estudio del aprendizaje de métricas de distancia, como en el software elaborado. Algunas de estas vías son:

\begin{itemize}
    \item \textbf{Incorporación de nuevos algoritmos al software elaborado.} Aunque el software desarrollado contiene los algoritmos clásicos más populares, y algunos más modernos, de esta disciplina, en la actualidad siguen surgiendo nuevas propuestas. Mantener el software al día con las nuevas propuestas será de interés para su consolidación.
    \item \textbf{Otros enfoques para el concepto de distancia.} La mayor parte del aprendizaje de métricas de distancia actual utiliza como distancias únicamente la familia de distancias de Mahalanobis. Sin embargo, en algunos artículos, se abre la puerta al aprendizaje de otras posibles distancias, como es el caso del aprendizaje local de distancias \cite{lmnn}. Desarrollando nuevos enfoques, dispondremos de una mayor variedad de distancias para aprender, teniendo así más posibilidades de éxito.
    \item \textbf{Kernelización de los algoritmos existentes.} La kernelización de algoritmos de aprendizaje de métricas de distancia puede extenderse a otros algoritmos, aparte de los presentados. La búsqueda de una parametrización adecuada y un teorema de representación que permita aplicar el kernel trick es otra posible tarea a llevar a cabo.
    \item \textbf{Otros mecanismos de optimización.} Los algoritmos estudiados optimizan sus funciones objetivo aplicando gradiente descendente, ya sean convexas o no. El uso de otras técnicas de optimización, utilizando por ejemplo metaheurísticas, puede ser útil a la hora de mejorar aquellos algoritmos que no tienen funciones objetivo convexas.
\end{itemize}