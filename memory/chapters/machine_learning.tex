\chapter{El aprendizaje automático}

En este capítulo se realiza una descripción general del aprendizaje automático, haciendo especial hincapié en el aprendizaje supervisado y los problemas de clasificación. El aprendizaje automático es la rama de las ciencias de la computación cuya finalidad es conseguir que los ordenadores aprendan de un conjunto de datos. Aprender en este contexto está relacionado con la identificación de patrones o la capacidad de generalización a nuevos datos. Veremos que hay diferentes modelos de aprendizaje, distinguiendo dos grandes grupos: supervisado y no supervisado. Profundizando en el aprendizaje supervisado, analizaremos los problemas de clasificación, sobre los que se centrarán la mayoría de las aplicaciones del problema que estudiaremos en los próximos capítulos.

\section{Introducción}



\section{El aprendizaje supervisado}



\section{El problema de clasificación}