\chapter*{Resumen}

En este trabajo se presenta la disciplina del aprendizaje de métricas de distancia, y se analizan teóricamente sus fundamentos, junto con los principales algoritmos de aprendizaje utilizados en esta disciplina. Además, se presenta un software que recoge todos los algoritmos estudiados, que ha sido utilizado además para elaborar diferentes experimentos sobre estos algoritmos.

En primer lugar se introducen los principales fundamentos matemáticos en los que se basa el aprendizaje de métricas de distancia. Estos fundamentos se presentan en tres bloques diferenciados. El primero de ellos es el análisis convexo, fundamental en la mayoría de las funciones a optimizar dentro de los algoritmos estudiados. En segundo lugar se presenta un estudio detallado de algunos aspectos de las matrices, haciendo especial hincapié en las matrices semidefinidas y la optimización por vectores propios. Este estudio será de gran importancia a la hora de modelar el problema. Por último, se presentan algunos conceptos de la teoría de la información, presentes en varios de los algoritmos estudiados.

En el análisis teórico del aprendizaje de métricas de distancia se recuerda, en primer lugar, el concepto de distancia, destacando las distancias de Mahalanobis, las cuales permitirán parametrizar los problemas a estudiar. A continuación se presenta el problema de aprendizaje, y cómo modelarlo, junto con algunas de sus principales aplicaciones. Finalmente se presenta el concepto de aprendizaje por semejanza, una familia de algoritmos de aprendizaje que utilizan distancias para lograr sus objetivos, y para las cuales el aprendizaje de métricas de distancia es de gran importancia. Finalmente se analizan teóricamente los algoritmos presentados, explicando su objetivo, la función que pretenden optimizar y los métodos de resolución que emplean.

Por último se presenta el software desarrollado, con sus principales características y ejemplos, y se describen los experimentos realizados, junto con los resultados obtenidos.

\textbf{Palabras clave:} aprendizaje de métricas de distancia, aprendizaje automático, clasificación, distancia de Mahalanobis, dimensionalidad, vecinos cercanos, semejanza, matrices, análisis convexo.