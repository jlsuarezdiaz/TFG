% **************************************************************************************************************
% A Classic Thesis Style
% An Homage to The Elements of Typographic Style
%
% Copyright (C) 2015 André Miede http://www.miede.de
%
% If you like the style then I would appreciate a postcard. My address 
% can be found in the file ClassicThesis.pdf. A collection of the 
% postcards I received so far is available online at 
% http://postcards.miede.de
%
% License:
% This program is free software; you can redistribute it and/or modify
% it under the terms of the GNU General Public License as published by
% the Free Software Foundation; either version 2 of the License, or
% (at your option) any later version.
%
% This program is distributed in the hope that it will be useful,
% but WITHOUT ANY WARRANTY; without even the implied warranty of
% MERCHANTABILITY or FITNESS FOR A PARTICULAR PURPOSE.  See the
% GNU General Public License for more details.
%
% You should have received a copy of the GNU General Public License
% along with this program; see the file COPYING.  If not, write to
% the Free Software Foundation, Inc., 59 Temple Place - Suite 330,
% Boston, MA 02111-1307, USA.
%
% **************************************************************************************************************
\RequirePackage{fix-cm} % fix some latex issues see: http://texdoc.net/texmf-dist/doc/latex/base/fixltx2e.pdf
\documentclass[ oneside,openany,titlepage,numbers=noenddot,headinclude,%1headlines,% letterpaper a4paper
                footinclude=true,cleardoublepage=empty,abstractoff, % <--- obsolete, remove (todo)
                BCOR=5mm,paper=a4,fontsize=11pt,%11pt,a4paper,%
                spanish,american%
                ]{scrreprt}

%********************************************************************
% Note: Make all your adjustments in here
%*******************************************************
\input{classicthesis-config}
\input{config}

%********************************************************************
% Bibliographies
%*******************************************************
\addbibresource{bibliography.bib}

%********************************************************************
% Hyphenation
%*******************************************************
%\hyphenation{put special hyphenation here}

% ********************************************************************
% GO!GO!GO! MOVE IT!
%*******************************************************
\begin{document}
\frenchspacing
\raggedbottom
\selectlanguage{spanish} % american ngerman
%\renewcommand*{\bibname}{new name}
%\setbibpreamble{}
\pagenumbering{roman}
\pagestyle{plain}
%********************************************************************
% Frontmatter
%*******************************************************
%\include{FrontBackmatter/DirtyTitlepage}
%\include{meta/Titlepage}
%\include{meta/Titleback}
%\cleardoublepage\include{meta/Abstract}
%\cleardoublepage\include{meta/AutorizacionBiblioteca}
%\cleardoublepage\include{meta/AutorizacionTutor}
%\cleardoublepage\include{FrontBackmatter/Foreword}
%\cleardoublepage\include{FrontBackmatter/Publications}
%\cleardoublepage\include{meta/Acknowledgments}
\pagestyle{scrheadings}
%\cleardoublepage\include{meta/Contents}
%********************************************************************
% Mainmatter
%*******************************************************
\cleardoublepage\pagenumbering{arabic}
%\setcounter{page}{90}
% use \cleardoublepage here to avoid problems with pdfbookmark
\cleardoublepage

\part{Introducción}

\chapter{Descripción}
%\input{chapters/intro.tex}

\chapter{Objetivos}
%\input{chapters/objetivos.tex}

\part{Matemáticas}

\chapter{Distancias y métricas}

\section{Distancias. Definición y ejemplos. Pseudodistancias.}

\subsection{Distancias}

\begin{definition}[Distancia]
	Sea $X$ un  conjuntono vacío. Una \emph{distancia} o \emph{métrica} definida sobre $X$ es una aplicación $d:X\times X \to \mathbb{R}^{+}_{0}$, verificando las siguientes propiedades:
	
	\begin{enumerate}[i)]
		\item $d(x,y)=0  \iff x = y \ \forall x,y \in X$ (Coincidencia)
		\item $d(x,y)=d(y,x) \ \forall x,y \in X$ (Simetría)
		\item $d(x,z)=d(x,y)+d(y,z) \ \forall x,y,z \in X$ (Desigualdad triangular)
	\end{enumerate}
	
	Al par ordenado $(X,d)$ se le denomina espacio métrico.
\end{definition}

\remb
Como consecuencia directa de la definición se tienen las siguientes propiedades:

\begin{enumerate}[i)]
	\setcounter{enumi}{3}
	\item $d(x,y) \ge 0 \ \forall x,y \in X$ (No negatividad)
	\item $|d(x,y)-d(y,z)| \le d(x,z) \ \forall x,y,z \in X$ (Buscar interpretación)
	\item $d(x_1,x_n) \le \sum_{i=1}^{n-1}d(x_i,x_{i+1}) \ \forall x_1,\dots,x_n \in E$ (Desigualdad triangular generalizada)
\end{enumerate}

\proofb
 $ $\newline
	\begin{enumerate}[i)]
		\setcounter{enumi}{3}
		\item $ 0 \us{i)}{=} \frac{1}{2}d(x,x) \us{iii)}{\le}\frac{1}{2}[d(x,y)+d(y,x)]\us{ii)}{=}d(x,y) \ \forall x,y \in X$
		
		\item Usando $ii)$ y $iii)$ se tiene:
		\[d(x,y) \le d(x,z) + d(z,y) = d(x,z)+d(y,z) \implies d(x,y)-d(y,z)\le d(x,z)\]
		\[d(y,z) \le d(y,x) + d(x,z) = d(x,y)+d(x,z) \implies d(y,z)-d(x,y)\le d(x,z) \]
		Por tanto podemos tomar valores absolutos en la diferencia obteniendo la desigualdad buscada.
		
		\item Es consecuencia de la desigualdad triangular aplicando inducción.
	\end{enumerate}
\proofe

\reme

Veamos algunos ejemplos de distancias:

\exampleb

\emph{Subespacios métricos}. Sea $(X,d)$ un espacio métrico y $A\subset X$. La aplicación $d_{|A}:A\times A \to \mathbb{R}^+_0$ es una distancia, y $(A,d_{|A})$ es un subespacio métrico de $(X,d)$.
	
\examplee

\exampleb

\emph{Distancia trivial}. Sea $X$ cualquier conjunto no vacío. Sobre $X$ definimos la aplicación $d:X\times X \to \mathbb{R}^+_0$ por:

\[d(x,y):= \begin{cases}
0 & \text{, si } x = y \\
1 & \text{, si } x \ne y
\end{cases}\]

$(X,d)$ es un espacio métrico, y $d$ es una distancia trivial que nos indica solo si dos elementos de $X$ son iguales o distintos.

\examplee

\exampleb

\emph{Distancia de Hamming}. Sean $(X_i,d_i), i=1,\dots, n$ espacios métricos con distancias triviales. Consideramos $X = X_1 \times \dots \times X_n$, $x = (x_1,\dots,x_n), y = (y_1,\dots,y_n) \in X$ y la aplicación $d:X\times X \to \mathbb{R}^+_0$ dada por

\[d(x,y) = \sum_{i=1}^n d(x_i,y_i)\]

La aplicación $d$ es una distancia que nos muestra el número de elementos que difieren entre dos vectores $x,y$ de $X$, y se conoce como distancia de Hamming. Es muy utilizada en algunos ámbitos de la teoría de la información.

\examplee

\exampleb

\emph{Distancias asociadas a normas}. Si $(X,\|.\|)$ es un espacio normado, se define la distancia asociada a la norma por $d(x,y)=\|x-y\| \ \forall x,y\in X$. Profundizaremos sobre estas distancias en la siguiente sección.

\examplee

\subsection{Pseudodistancias}

El concepto de distancia se puede suavizar, relajando la condición de coincidencia, obteniendo así lo que se conoce como una pseudodistancia, una aplicación que mantiene muchas de las propiedades de una distancia, y en muchos campos, como el que vamos a tratar, puede aplicarse con la misma utilidad que las distancias. Veamos su definición y algunos ejemplos y propiedades.

\begin{definition}[Distancia]
	Sea $X$ un conjunto no vacío. Una \emph{pseudodistancia} definida sobre $X$ es una aplicación $d:X\times X \to \mathbb{R}^{+}_{0}$, verificando las siguientes propiedades:
	
	\begin{enumerate}[i)]
		\item $d(x,x)=0  \ \forall x \in X$ 
		\item $d(x,y)=d(y,x) \ \forall x,y \in X$ (Simetría) 
		\item $d(x,z)=d(x,y)+d(y,z) \ \forall x,y,z \in X$ (Desigualdad triangular) 
	\end{enumerate}
	
\end{definition}

\remb

Podemos ver que el único cambio de la definición consiste en eliminar una de las implicaciones en la propiedad $i)$ (ahora puede haber elementos distintos con distancia nula entre ellos). Este cambio no afecta a la demostración de las propiedades $iv)$, $v)$ y $vi)$ de la distancia, luego estas siguen siendo válidas en las pseudodistancias.

\reme

Veamos algunos ejemplos de pseudodistancias:

\exampleb

\emph{Ejemplos básicos:}

\begin{enumerate}[1]
	\item Toda distancia sobre $X$ es una pseudodistancia sobre $X$.
	\item La aplicación nula $d:X\times X \to \mathbb{R}^+_0$ dada por $d(x,y) = 0 \ \forall x,y \in X$ es una pseudodistancia.
	
\end{enumerate}

\examplee

\exampleb

\emph{Espacios de funciones integrables.} Sea $\Omega \subset \mathbb{R}$ y consideramos, para $1 < p < \infty$, los espacios de funciones integrables 
\[L^p(\Omega)=\left\{f:\Omega \to \mathbb{R} : f \text{ es medible y } \int_{\Omega} |f(t)|^p dt < \infty \right\}.\]

Dadas $f,g \in L^p(\Omega)$ definimos la pseudodistancia entre ellas como \[d(f,g)=\left(\int_{\Omega}|f(t)-g(t)|^p dt\right)^{1/p}.\]

Es claro que se verifican las propiedades $i)$ y $ii)$ de pseudodistancia, y la $iii)$ es una aplicación directa de la desigualdad integral de Minkowski. Sin embargo, no es una distancia puesto que si $d(f,g)=0$ solo tenemos asegurada la igualdad casi por doquier.

\examplee

\exampleb

\emph{Pseudodistancias asociadas a seminormas}. Si $X$ es un espacio vectorial y $\|.\|$ es una seminorma, se define la distancia asociada a la seminorma por $d(x,y)=\|x-y\| \ \forall x,y\in X$. Profundizaremos en ellas en la siguiente sección.

\examplee

Para concluir esta sección, vamos a mostrar que a partir de una pseudodistancia podemos definir una relación de equivalencia mediante la cual, tras identificar los elementos en las mismas clases, podemos obtener un espacio métrico.

\propb

Sea $X$ un conjunto no vacío y $d:X\times X \to \mathbb{R}^+_0$ una pseudodistancia sobre $X$. Definimos la relación $x \sim y \iff d(x,y)=0$. $\sim$ es una relación de equivalencia.

\prope

\proofb

$ $ \newline

\begin{itemize}
	
\item \emph{Reflexiva.} Consecuencia de la propiedad $i)$ de pseudodistancia.
\item \emph{Simétrica.} Consecuencia de la propiedad $ii)$ de pseudodistancia.
\item \emph{Transitiva.} Consecuencia de la propiedad $iii)$ de pseudodistancia.
\end{itemize}

\proofe

\theob

En las condiciones anteriores, el cociente $X/_\sim$ es un espacio métrico con la distancia $\hat{d}:X/_\sim \times X/_\sim \to \mathbb{R}^+_0$ dada por $\hat{d}([x],[y])=d(x,y) \ \forall[x],[y] \in X/_\sim$

\theoe

\proofb

En primer lugar veamos que la aplicación $\hat{d}$ está bien definida. Para ello veamos que la distancia no depende del representante escogido. Supongamos $[x]=[x']$ y $[y]=[y']$ (lo que implica que $d(x,x')=0=d(y,y')$. Queremos ver que $d(x,y)=d(x',y')$. Aplicamos varias veces la desigualdad triangular.

\[d(x,y)\le \ub{d(x,x')}_{=0}+d(x',y) \le d(x',y')+\ub{d(y',y)}_{=0}\]
\[\le \ub{d(x',x)}_{=0}+d(x,y') \le d(x,y) + \ub{d(y,y')}_{=0} \]

Por tanto, $d(x,y) \le d(x',y') \le d(x,y)$, obteniendo la igualdad. Que $\hat{d}$ es una distancia es inmediato por la definición de la relación de equivalencia y las propiedades de $d$.

\proofe

De los ejemplos anteriores podemos obtener los primeros espacios métricos a partir de cocientes (los asociados a seminormas los veremos en la siguiente sección):

\begin{itemize}
	\item Si $(X,d)$ es un espacio métrico, la relación de equivalencia es la igualdad y el espacio cociente es esencialmente idéntico a $(X,d)$.
	\item Para cualquier conjunto $X$ no vacío, la pseudodistancia nula origina el espacio cociente de un solo punto, donde la aplicación nula sí es una distancia.
	\item Los espacios cociente de los $L^p$ bajo la relación dada por la pseudodistancia anterior (en este caso es la igualdad c.p.d.) son los conocidos espacios de Banach de funciones integrables $\mathcal{L}^p$
\end{itemize}
 
\section{Distancias en espacios euclídeos. Distancia de Mahalanobis.}




\chapter{Optimización matricial}


\part{Informática}

\chapter{Aprendizaje automático}\label{ch:learning}


\chapter{Aprendizaje por semejanza. Aprendizaje de métricas}\label{ch:sim}

\chapter{Algoritmos de aprendizaje de métricas}

\section{PCA}

El \emph{Análisis de Componentes Principales (PCA, Principal Component Analysis)} es una técnica cuya finalidad es transformar los datos de forma que cada atributo de los datos transformados permita explicar la mayor varianza posible, dentro de la que quede disponible.


\part{Conclusiones}

\chapter{Conclusiones y vías futuras}\label{ch:conclusions}


% ********************************************************************
% Backmatter
%*******************************************************
\appendix
%\renewcommand{\thechapter}{\alph{chapter}}
\cleardoublepage
%\part{Apéndice}
%\chapter{Manual de usuario de Ruta}\label{ch:manual}
%\input{chapters/manual.tex}
%\include{Chapters/Chapter0A}
%********************************************************************
% Other Stuff in the Back
%*******************************************************
%\cleardoublepage\include{meta/Bibliography}
%\cleardoublepage\include{FrontBackmatter/Declaration}
%\cleardoublepage\include{FrontBackmatter/Colophon}
% ********************************************************************
% Game Over: Restore, Restart, or Quit?
%*******************************************************
\end{document}
% ********************************************************************
