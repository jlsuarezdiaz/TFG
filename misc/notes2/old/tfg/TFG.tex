%%%%%%%%%%%%%%%%%%%%%%%%%%%%%%%%%%%%%%%%%%%%%%%%%%%%%%%%%%%%%%%%%%%%%%%%%%%%%%%%%%%%%%%%%%%%%%%%%%%%%%
% Plantilla básica de Latex en Español.
%
% Autor: Andrés Herrera Poyatos (https://github.com/andreshp)
%
% Es una plantilla básica para redactar documentos. Utiliza el paquete fancyhdr para darle un
% estilo moderno pero serio.
%
% La plantilla se encuentra adaptada al español.
%
%%%%%%%%%%%%%%%%%%%%%%%%%%%%%%%%%%%%%%%%%%%%%%%%%%%%%%%%%%%%%%%%%%%%%%%%%%%%%%%%%%%%%%%%%%%%%%%%%%%%%%

%-----------------------------------------------------------------------------------------------------
%	INCLUSIÓN DE PAQUETES BÁSICOS
%-----------------------------------------------------------------------------------------------------

\documentclass{article}

\usepackage{lipsum}                     % Texto dummy. Quitar en el documento final.

\usepackage{biblatex}

\addbibresource{bibliography.bib}

\usepackage{spanish}

\usepackage{mathematics}

\usepackage{comment}


%-----------------------------------------------------------------------------------------------------
%	SELECCIÓN DEL LENGUAJE
%-----------------------------------------------------------------------------------------------------

% Paquetes para adaptar Látex al Español:
\usepackage[spanish,es-noquoting, es-tabla, es-lcroman]{babel} % Cambia
\usepackage[utf8]{inputenc}                                    % Permite los acentos.
\selectlanguage{spanish}                                       % Selecciono como lenguaje el Español.

%-----------------------------------------------------------------------------------------------------
%	SELECCIÓN DE LA FUENTE
%-----------------------------------------------------------------------------------------------------

% Fuente utilizada.
\usepackage{courier}                    % Fuente Courier.
\usepackage{microtype}                  % Mejora la letra final de cara al lector.

%-----------------------------------------------------------------------------------------------------
%	LICENCIA
%-----------------------------------------------------------------------------------------------------

\usepackage[
type={CC},
modifier={by},
version={4.0}, 
]{doclicense}

%-----------------------------------------------------------------------------------------------------
%	ESTILO DE PÁGINA
%-----------------------------------------------------------------------------------------------------

% Paquetes para el diseño de página:
\usepackage{fancyhdr}               % Utilizado para hacer títulos propios.
\usepackage{lastpage}               % Referencia a la última página. Utilizado para el pie de página.
\usepackage{extramarks}             % Marcas extras. Utilizado en pie de página y cabecera.
\usepackage[parfill]{parskip}       % Crea una nueva línea entre párrafos.
\usepackage{geometry}               % Asigna la "geometría" de las páginas.

% Se elige el estilo fancy y márgenes de 3 centímetros.
\pagestyle{fancy}
\geometry{left=3cm,right=3cm,top=3cm,bottom=3cm,headheight=1cm,headsep=0.5cm} % Márgenes y cabecera.
% Se limpia la cabecera y el pie de página para poder rehacerlos luego.
\fancyhf{}

% Espacios en el documento:
\linespread{1.1}                        % Espacio entre líneas.
\setlength\parindent{0pt}               % Selecciona la indentación para cada inicio de párrafo.

% Cabecera del documento. Se ajusta la línea de la cabecera.
\renewcommand\headrule{
	\begin{minipage}{1\textwidth}
	    \hrule width \hsize
	\end{minipage}
}

% Texto de la cabecera:
\lhead{\docauthor}                          % Parte izquierda.
\chead{}                                    % Centro.
\rhead{\subject \ - \doctitle}              % Parte derecha.

% Pie de página del documento. Se ajusta la línea del pie de página.
\renewcommand\footrule{
\begin{minipage}{1\textwidth}
    \hrule width \hsize
\end{minipage}\par
}

\lfoot{}                                                 % Parte izquierda.
\cfoot{}                                                 % Centro.
\rfoot{Página\ \thepage\ de\ \protect\pageref{LastPage}} % Parte derecha.


%-----------------------------------------------------------------------------------------------------
% DEFINICIÓN DE COMANDOS
%-----------------------------------------------------------------------------------------------------

\newcommand{\istargetof}{\rightarrow}

%-----------------------------------------------------------------------------------------------------
%	PORTADA
%-----------------------------------------------------------------------------------------------------

% Elija uno de los siguientes formatos.
% No olvide incluir los archivos .sty asociados en el directorio del documento.
\usepackage{title1}
%\usepackage{title2}

%-----------------------------------------------------------------------------------------------------
%	TÍTULO, AUTOR Y OTROS DATOS DEL DOCUMENTO
%-----------------------------------------------------------------------------------------------------

% Título del documento.
\newcommand{\doctitle}{Trabajo de fin de grado}
% Subtítulo.
\newcommand{\docsubtitle}{El aprendizaje de métricas de distancia: análisis y revisión de técnicas desarrolladas en alta dimensionalidad}
% Fecha.
\newcommand{\docdate}{\today}
% Asignatura.
\newcommand{\subject}{}
% Autor.
\newcommand{\docauthor}{Juan Luis Suárez Díaz}
\newcommand{\docaddress}{Universidad de Granada}
\newcommand{\docemail}{jlsuarezdiaz@correo.ugr.es}

%-----------------------------------------------------------------------------------------------------
%	RESUMEN
%-----------------------------------------------------------------------------------------------------

% Resumen del documento. Va en la portada.
% Puedes también dejarlo vacío, en cuyo caso no aparece en la portada.
%\newcommand{\docabstract}{}
\newcommand{\docabstract}{}

\begin{document}

\maketitle

%-----------------------------------------------------------------------------------------------------
%	ÍNDICE
%-----------------------------------------------------------------------------------------------------

% Profundidad del Índice:
%\setcounter{tocdepth}{1}

\newpage
\tableofcontents
\newpage

%-----------------------------------------------------------------------------------------------------
%	SECCIÓN 1
%-----------------------------------------------------------------------------------------------------

\part{Matemáticas}

\part{Informática teórica}

\chapter{El aprendizaje automático}

\chapter{El aprendizaje de métricas de distancia}

\chapter{Descripción teórica de las técnicas de aprendizaje de métricas de distancia}

\section{Técnicas de reducción de dimensionalidad}

\subsection{PCA}

TODO

\subsection{LDA}

TODO

\subsection{ANMM}

TODO

\section{Técnicas orientadas a la mejora del clasificador de vecinos cercanos}

\subsection{LMNN}

LMNN (\textit{Large Margin Nearest Neighbors}) \cite{lmnn} es un algoritmo de aprendizaje de métricas de distancia orientado específicamente a mejorar la precisión del clasificador kNN. Se basa en la premisa de que el kNN clasificará con más fiabilidad un ejemplo si sus $k$ vecinos comparten la misma etiqueta, y para ello intenta aprender una distancia que maximice el número de ejemplos que comparten etiqueta con el mayor número de vecinos posible.

De esta forma, el algoritmo LMNN trata de minimizar una función de error que penaliza, por un lado, las distancias grandes entre cada ejemplo y los considerados como sus vecinos ideales, y por otro lado, las distancias pequeñas entre ejemplos de distintas clases.

Supongamos que tenemos un conjunto de datos $\mathcal{X} = \{x_1,\dots,x_N\} \subset \mathbb{R}^d$ con etiquetas $\mathcal{Y} = \{y_1,\dots,y_d\}$. Para su funcionamiento, el algoritmo hace uso del concepto de \emph{vecinos objetivo} o \emph{target neighbors}. Dado un ejemplo $x_i \in \mathcal{X}$, sus $k$ vecinos objetivos son aquellos ejemplos de la misma clase que $x_i$, y distintos de este, para los que se desea que sean considerados como vecinos en la clasificación del kNN. Si $x_j$ es un vecino objetivo de $x_i$, entonces lo notaremos $j \istargetof i$. Estos vecinos objetivo están fijos durante el proceso de aprendizaje. Si se dispone de alguna información a priori se puede utilizar para determinarlos. En caso contrario, una buena opción es utilizar los vecinos cercanos de la misma clase para la distancia euclídea.

Una vez establecidos los vecinos objetivo, para cada distancia y para cada ejemplo que manejemos podemos establecer un perímetro determinado por el vecino más lejano a dicho ejemplo. Buscamos distancias para las cuales no haya ejemplos de otras clases en dicho perímetro. Hay que destacar que con este perímetro no hay suficientes garantías de separación, pues la distancia encontrada podría haber colapsado todos los vecinos objetivo en un punto y entonces el perímetro tendría radio cero. Por ello se considera un margen determinado por el radio del perímetro, al que se añade una constante positiva. Veremos que no hay pérdida de generalidad, por la función objetivo que vamos a definir, en suponer que dicha constante es 1. A cualquier ejemplo de distinta clase que invada este margen lo llamaremos \emph{impostor}. Nuestro objetivo, por tanto, será, además de acercar cada ejemplo a sus vecinos objetivo lo máximo posible, intentar alejar lo máximo posible a los impostores.

TODO

\subsection{NCA}

TODO

\section{Técnicas orientadas a la mejora del clasificador de centroides cercanos}

\subsection{NCMML}

TODO

\subsection{NCMC}

TODO


\section{Técnicas basadas en teoría de la información}

\subsection{ITML}

TODO

\subsection{DMLMJ}

TODO

\subsection{MCML}

TODO

\section{Otras técnicas de aprendizaje de métricas de distancia}

\subsection{LSI}

TODO

\subsection{DML-eig}

TODO

\subsection{LDML}

TODO

\section{El kernel trick. Algoritmos de aprendizaje de métricas de distancia basados en kernels}

\subsection{El kernel trick}

TODO

\subsection{KLMNN}

TODO

\subsection{KANMM}

TODO

\subsection{KDMLMJ}

TODO

\subsection{KDA}

TODO


\newpage
\printbibliography


\end{document}
