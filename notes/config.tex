%% Stuff from Pandoc
\providecommand{\tightlist}{%
  \setlength{\itemsep}{0pt}\setlength{\parskip}{0pt}}

\setlength{\parskip}{1mm}
%\setlength{\mathindent}{0mm}

%% Algorithms
\usepackage{algorithm}
\usepackage{algorithmic}
\usepackage{enumerate}
\renewcommand{\algorithmicrequire}{\textbf{Entrada:}}
\renewcommand{\algorithmicensure}{\textbf{Salida:}}
\makeatletter
\renewcommand{\ALG@name}{Algoritmo}
\renewcommand{\listalgorithmname}{Lista de \ALG@name s}
\makeatother

%% Tikz for neural network diagrams
\usepackage{tikz}

%% Theorem environments
\usepackage{aliascnt}
\def\NewTheorem#1#2{%
  \newaliascnt{#1}{theorem}
  \newtheorem{#1}[#1]{#2}
  \aliascntresetthe{#1}
  \expandafter\def\csname #1autorefname\endcsname{#2}
}
\usepackage{amsthm}
% \newtheorem{satz}{Satz}[chapter]
% \newtheorem*{satz*}{Satz}
% \newtheorem{lemma}[satz]{Lemma}
% \newtheorem{corollar}[satz]{Korollar} 
% \newcommand{\satzautorefname}{Satz}
\newtheorem{theorem}{Teorema}[chapter]
\NewTheorem{lemma}{Lema}
\NewTheorem{prop}{Proposición}
\NewTheorem{cor}{Corolario}
%\newtheorem{lemma}[theorem]{Lema}
%\newtheorem{prop}[theorem]{Proposición}
%\newtheorem{cor}[theorem]{Corolario}

\theoremstyle{definition}
\newtheorem{definition}{Definición}[chapter]
\newtheorem{example}{Ejemplo}[chapter]
\newtheorem{exca}{Ejercicio}[chapter]

\theoremstyle{remark}
\newtheorem{remark}{Observación}[chapter]

% Autoref commands
%\newcommand{\corollaryautorefname}{corolario}
%\newcommand{\corautorefname}{corolario}
%\newcommand{\propautorefname}{proposición}
%\newcommand{\propositionautorefname}{proposición}
%\newcommand{\lemmaautorefname}{lema}
\newcommand*{\definitionautorefname}{definición}
\newcommand*{\exampleautorefname}{ejemplo}
\newcommand*{\remarkautorefname}{observación}
\newcommand*{\algorithmautorefname}{algoritmo}
\def\theoremautorefname{teorema}
\def\sectionautorefname{sección}

%\numberwithin{equation}{section}

% Replacing environments by pairs of commands for use in Markdown
\newcommand{\defineb}{\begin{definition}}
\newcommand{\definee}{\end{definition}}
\newcommand{\theob}{\begin{theorem}}
\newcommand{\theoe}{\end{theorem}}
\newcommand{\lemmab}{\begin{lemma}}
\newcommand{\lemmae}{\end{lemma}}
\newcommand{\propb}{\begin{prop}}
\newcommand{\prope}{\end{prop}}
\newcommand{\remb}{\begin{remark}}
\newcommand{\reme}{\end{remark}}
\newcommand{\proofb}{\begin{proof}}
\newcommand{\proofe}{\end{proof}}
\newcommand{\exampleb}{\begin{example}}
\newcommand{\examplee}{\end{example}}
\newcommand{\corb}{\begin{cor}}
\newcommand{\core}{\end{cor}}

% hyperref
%\PassOptionsToPackage{xetex,hyperfootnotes=false,pdfpagelabels}{hyperref}
%    \usepackage{hyperref}

%% ATAJOS
\newcommand{\RR}{\mathbb{R}}
\newcommand{\NN}{\mathbb{N}}
\newcommand{\ZZ}{\mathbb{Z}}
\newcommand{\KK}{\mathbb{K}}
\newcommand{\LL}{\mathcal{L}}

% Distribuciones de probabilidad
\newcommand{\PN}{\mathcal{N}}

% Traspuesta
\newcommand{\Tr}[1]{#1^{\mathrm{T}}}
% Norma
\newcommand{\norm}[1]{\left\lVert#1\right\rVert}

%\newcommand{\dim}{\mathrm{dim}}
\newcommand{\E}{\mathrm{E}}
\let\Pr\relax
\newcommand{\Pr}[1]{\mathrm{P}\left[#1\right]}
\newcommand{\Var}{\mathrm{Var}}
\newcommand{\Ker}{\mathrm{Ker}}
\newcommand{\Mid}{\mid\mid}

\newcommand{\asconv}{\overset{cs}{\rightarrow}}
\newcommand{\pconv}{\overset{P}{\rightarrow}}
\newcommand{\softmax}[1]{\mathrm{softmax}(#1)}

\newcommand{\us}[2]{\underset{#1}{#2}}
\newcommand{\ub}[1]{\underbrace{#1}}

\newcommand{\note}[1]{\paragraph{Nota}#1}